%%%%%%%%%%%%%%%%%%%%%%%%%%%%%%%%%%%%%%%%%
% Twenty Seconds Resume/CV
% LaTeX Template
% Version 1.1 (8/1/17)
%
% This template has been downloaded from:
% http://www.LaTeXTemplates.com
%
% Original author:
% Carmine Spagnuolo (cspagnuolo@unisa.it) with major modifications by 
% Vel (vel@LaTeXTemplates.com)
%
% License:
% The MIT License (see included LICENSE file)
%
%%%%%%%%%%%%%%%%%%%%%%%%%%%%%%%%%%%%%%%%%

%----------------------------------------------------------------------------------------
%	PACKAGES AND OTHER DOCUMENT CONFIGURATIONS
%----------------------------------------------------------------------------------------

\documentclass[letterpaper]{template/twentysecondcv} % a4paper for A4



\hyphenation{elektro-ener-ge-tycznym}
\hyphenation{wielo-mo-do-we}
\hyphenation{sze-re-go-we}
%----------------------------------------------------------------------------------------
%	 PERSONAL INFORMATION
%----------------------------------------------------------------------------------------

% If you don't need one or more of the below, just remove the content leaving the command, e.g. \cvnumberphone{}

\profilepic{img/photo2.jpg} % Profile picture

\cvname{mgr inż. Jakub Głuszek} % Your name
\cvjobtitle{Programista/Automatyk} % Job title/career

\cvdate{28 marca 1994} % Date of birth
\cvaddress{Polska, Gdańsk} % Short address/location, use \newline if more than 1 line is required
\cvnumberphone{} % Phone number
%\cvsite{http://en.wikipedia.org} % Personal website
\cvmail{jjgluszek@gmail.com} % Email address

%----------------------------------------------------------------------------------------

\begin{document}

%----------------------------------------------------------------------------------------
%	 ABOUT ME
%----------------------------------------------------------------------------------------

\aboutme{Ambitny programista/automatyk zawodowo pracujący w sektorze energetyki. Kierownik projektów układów automatyki do regulacji stacji transformatorowych. Programista aplikacji desktopowych pisanych w~językach wysokiego poziomu. Freelancer utrzymujący i~rozwijający webową aplikację ASP.NET MVC.} % To have no About Me section, just remove all the text and leave \aboutme{}

%----------------------------------------------------------------------------------------
%	 SKILLS
%----------------------------------------------------------------------------------------

% Skill bar section, each skill must have a value between 0 an 6 (float)
\skills{
{Git/4.5},
{\LaTeX/4.5},
{Embedded ARM(STM32)/4.0},
{SQL/3.0},
{HTML,CSS/3.0},
{C,C++/3.0},
{MATLAB/5.0},
{ASP.NET/4.0},
{{.NET (C\#)}/5.0}}

\languages{{Angielski/Biegła znajomość w mowie i piśmie}, {Polski/Język ojczysty}}

%------------------------------------------------

% Skill text section, each skill must have a value between 0 an 6
%\skillstext{{lovely/4},{narcissistic/3}}

%----------------------------------------------------------------------------------------

\makeprofile % Print the sidebar


%----------------------------------------------------------------------------------------
%	 EDUCATION
%----------------------------------------------------------------------------------------
\vspace{-0.25cm}
\section{Edukacja}

\begin{twenty} % Environment for a list with descriptions
	\twentyitem{2017-2018}{Politechnika Gdańska}{}{Studia magisterskie na wydziale Elektroniki, Telekomunikacji i Informatyki na kierunku Automatyka i Robotyka.}
	\twentyitem{2013-2017}{Politechnika Gdańska}{}{Studia inżynierskie na wydziale Elektroniki, Telekomunikacji i Informatyki na kierunku Automatyka i Robotyka. Członek koła naukowego SKALP.}
	%\twentyitem{2010-2013}{XX Liceum Ogólnokształcące w Gdańsku}{}{Profil matematyczno-fizyczny. Uczestnictwo w samorządowym projekcie ,,Zdolni z Pomorza'' w dziedzinie matematyki.}
	%\twentyitem{<dates>}{<title>}{<location>}{<description>}
\end{twenty}

%\begin{twenty}
%\twentyitem{1234}{lalal}{lalala}{lalalalalalla}
%\end{twenty}

%----------------------------------------------------------------------------------------
%	 PUBLICATIONS
%----------------------------------------------------------------------------------------

%\section{Publications}

%\begin{twentyshort} % Environment for a short list with no descriptions
%	\twentyitemshort{1865}{Chapter One, Down the Rabbit Hole.}
%	\twentyitemshort{1865}{Chapter Two, The Pool of Tears.}
%	\twentyitemshort{1865}{Chapter Three,  The Caucus Race and a Long Tale.}
%	\twentyitemshort{1865}{Chapter Four,  The Rabbit Sends a Little Bill.}
%	\twentyitemshort{1865}{Chapter Five,  Advice from a Caterpillar.}
	%\twentyitemshort{<dates>}{<title/description>}
%\end{twentyshort}

%----------------------------------------------------------------------------------------
%	 AWARDS
%----------------------------------------------------------------------------------------
\vspace{-0.25cm}
\section{Nagrody}

\begin{twentyshort} % Environment for a short list with no descriptions
	\twentyitem{2015}{III miejsce w hackathonie STM32 ,,Touch the World''}{} {Projekt ,,Miniaturowego zdalnie sterowanego pojazdu badawczego".}
	%\twentyitemshort{1998}{All-Time Best Fantasy Novel before 1990.}
	%\twentyitemshort{<dates>}{<title/description>}
\end{twentyshort}

%----------------------------------------------------------------------------------------
%	 EXPERIENCE
%----------------------------------------------------------------------------------------
\vspace{-0.25cm}
\section{Doświadczenie zawodowe}

\begin{twenty} % Environment for a list with descriptions
	\twentyitem{2015\small{-obecnie}}{Instytut Energetyki Instytut Badawczy O/Gdańsk}{}{Praca na stanowisku specjalisty inżynieryjno-technicznego. Do obowiązków należy zaliczyć prowadzenie projektów, konstruowanie urządzeń zarówno od strony sprzętowej jak i~oprogramowania, pisanie aplikacji w językach wysokiego poziomu.}
	\twentyitem{2019\small{-obecnie}}{Wyższa Szkoła Bankowa w Gdańsku}{}{Prowadzenie zajęć  laboratoryjnych z przedmiotów: programowanie obiektowe, zaawansowane techniki obiektowe, programowanie w~C++ oraz postawy przetwarzania sygnałów.}
\end{twenty}

%----------------------------------------------------------------------------------------
%	 OTHER INFORMATION
%----------------------------------------------------------------------------------------
\vspace{-0.25cm}
\section{Wybrane projekty}
\subsection{Zawodowe}
\vspace{-0.25cm}


\subsubsection*{Układ regulacji nadrzędnej dla farm wiatrowych}
Projekt aplikacji desktopowej WPF (MVVM) .NET (C\#), która steruje nadrzędnym regulatorem farmy wiatrowej. W projekcie została wykorzystana m.in.: komunikacja z serwerem OPC oraz łączność z ośrodkami nadrzędnymi w oparciu o stos TCP/IP.

\subsubsection*{Utrzymanie i rozwój systemu CRM dla firmy telekomunikacyjnej}
Utrzymanie i rozwój aplikacji opartej na platformie .NET pisanej w wykorzystaniem framework'a ASP.NET MVC dla jednej z wiodących firm telekomunikacyjnych w Gdańsku. 

\subsubsection*{Układy automatycznej regulacji stacji transformatorowych}
Kierowanie realizacją projektów automatyki ARST dla stacji transformatorowych. W~tym sporządzaniem dokumentacji technicznych, przygotowywaniem list sygnałów, testów FAT jak i późniejszym wdrożeniem układów na obiekcie elektroenergetycznym.

\subsubsection*{Przetwornik numeru zaczepu transformatora}
Konstrukcja urządzenia przetwarzającego aktualny stan zaczepu transformatora\\ i~przekazującego tę informację do innych urządzeń. W projekcie poza standardowymi wyjściami dwustanowymi zostały wykorzystane protokoły komunikacyjne szeregowe oraz sieciowe. Przetwornik posiada wyjścia światłowodowe wielomodowe, Ethernet, RS485 oraz pętlę prądową 4-20 mA.

\subsubsection*{Konwerter pętli prądowej $\pm$~20 mA na protokół Modbus}
Układ przetwarzający szeregową komunikację zrealizowaną na pętli prądowej (cztery kanały) na protokół Modbus (dwa kanały) wykorzystujący interfejs RS485.

\subsection{Hobbistyczne}
\vspace{-0.25cm}
\subsubsection*{Cyfrowy przedwzmacniacz gramofonowy RIAA}
Skonstruowany z oparciu do mikrokontroler STM32F7. Układ realizuje korekcję RIAA. W projekcie został wykorzystany m.in.: koprocesor DSP, wyświetlacz dotykowy, zewnętrzną pamięć SDRAM i kodeki audio. 


%----------------------------------------------------------------------------------------
%	 INTERESTS
%----------------------------------------------------------------------------------------

%\section{Zainteresowania}
%
%Szeroko rozumiana elektronika i robotyka, w tym budowa własnego drona od warstwy sprzętowej po oprogramowanie. 

\vspace{0.9cm}
\tiny{Wyrażam zgodę na przetwarzanie moich danych osobowych dla potrzeb niezbędnych do realizacji procesu rekrutacji zgodnie z Rozporządzeniem Parlamentu Europejskiego i Rady (UE) 2016/679 z dnia 27 kwietnia 2016 r. w sprawie ochrony osób fizycznych w związku z przetwarzaniem danych osobowych i w sprawie swobodnego przepływu takich danych oraz uchylenia dyrektywy 95/46/WE (RODO).}


\end{document} 
